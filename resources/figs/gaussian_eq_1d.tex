\documentclass[xcolor={dvipsnames,rgb}, preview, aspectratio = 1609]{beamer}
% xcolor: e.g. provides \colorlet
% dvipsnames option: extra named colours
% aspectratio=1610: for modern widescreen monitors...
% see https://www.overleaf.com/learn/latex/Using_colours_in_LaTeX for more detail

% equation annotations also work in articles, of course (then might need explicit 
\usepackage[dvipsnames]{xcolor}

%%% standard math packages for equations:
\usepackage{amsmath}
\usepackage{amssymb}
\usepackage{mathtools}

\usepackage{annotate-equations}

%gets rid of bottom navigation bars
\setbeamertemplate{footline}[frame number]{}

%gets rid of bottom navigation symbols
\setbeamertemplate{navigation symbols}{}

%gets rid of footer
%will override 'frame number' instruction above
%comment out to revert to previous/default definitions
\setbeamertemplate{footline}{}

\begin{document}
    \colorlet{colorp}{NavyBlue}
    \colorlet{colorT}{WildStrawberry}
    \colorlet{colork}{OliveGreen}
    \colorlet{colorM}{RoyalPurple}
    \colorlet{colorNb}{Plum}
    \colorlet{colorIs}{black}
\begin{frame}{}
    \Large %%% increase font size to make equation more readable!

    %%% we define our own colors upfront - this makes it easier to keep it consistent if you change your mind

%    
    
    \begin{equation*}
    	p(x|\eqnmark[black]{mu}{\mu}, \eqnmark[black]{sigma}{\sigma^{2}}) = \eqnmarkbox[NavyBlue]{const}{\frac{1}{\sigma\sqrt{2\pi}}} \,\eqnmarkbox[Plum]{exp}{\exp}\left\{{\eqnmarkbox[OliveGreen]{neg}{-}\frac{1}{2} \eqnmarkbox[WildStrawberry]{dist}{\left(\frac{x-\mu}{\sigma}\right)^{2}}}\right\}
%    \mathcal{O}\big(
%        (
%        \eqnmarkbox[NavyBlue]{p1}{p}
%        \eqnmarkbox[OliveGreen]{k1}{\kappa}^3  % note that we have the ^3 outside the \eqnmark/\tikzmarknode arguments
%        )
%        \eqnmarkbox[WildStrawberry]{T1}{T}
%        +
%        (
%        \eqnmarkbox[NavyBlue]{p2}{p}  % tikz nodes need distinct names!
%        \eqnmark[OliveGreen]{k2}{\kappa}
%        )
%        (
%        \eqnmarkbox[WildStrawberry]{T2}{T}^2
%        \tikzmarknode{Is}{|\mathcal{I}^*|}  % manual \tikzmarknode works, too
%        \eqnmarkbox[Plum]{Nb}{N_b}
%        \eqnmark[RoyalPurple]{M}{M}
%        )
%    \big)
\end{equation*}
%
%\annotatetwo[yshift=1em]{above}{p1}{p2}{\# of nodes}
%\annotatetwo[yshift=-1em,xshift=0.2ex]{below}{T1}{T2}{\# of graphs in $\hat{\mathcal{G}}_T$}
%\annotatetwo[yshift=-2em]{below}{k1}{k2}{max.\ indegree in $\hat{\mathcal{G}}_T$}
%\annotate[yshift=2em]{above,left}{Is}{size of set of allowed interventions}
%\annotate[yshift=1em]{above}{Nb}{\# of samples per batch}
%\annotate[yshift=-1em]{below}{M}{\# of samples for $\mathbb{E}_y$}
%    \begin{tikzpicture}
\annotate[yshift=-1em]{below,left}{mu}{mean}
\annotate[yshift=-1em]{below,right}{sigma}{variance}
\annotate[yshift=1em]{above,left}{const}{normalising constant}
\annotate[yshift=1em]{above,right}{dist}{distance between $x$ and $\mu$}
\annotate[yshift=-2.5em]{below,left}{exp}{make sure $p(x) >0$}
\annotate[yshift=-2.5em]{below,right}{neg}{inversely related to $p(x)$}

%\end{tikzpicture}
%
%    
%    \begin{equation*}
%    	\eqnmarkbox[blue]{node1}{e_q^n}
%		\eqnmark[red]{node2}{f(x)}
%		\tikzmarknode{node3}{kT}
%    \end{equation*}
%	\annotate[yshift=1em]{}{node1,node2}{my annotation text}
%    \begin{equation*}
%        \color{BurntOrange}
%        \mathcal{O}\big(
%            (
%            % \tikzmarknode is what links parts of the equation and corresponding annotations
%            \eqnmarkbox[colorp]{p1}{p}
%            \eqnmarkbox[colork]{k1}{\kappa}^3 % note that we have the ^3 outside the \tikzmarknode
%            )
%            \eqnmarkbox{T1}{T}
%            +
%            (
%            \eqnmarkbox[colorp]{p2}{p} % tikzmarks need distinct names!
%            \eqnmark[colork]{k2}{\kappa}
%            )
%            (
%            \eqnmarkbox[colorT]{T2}{T}^2
%            \tikzmarknode{Is}{|\mathcal{I}^*|}
%            \eqnmarkbox[colorNb]{Nb}{N}_{\!\!\eqnmarkbox[BurntOrange]{b}{b}}
%            \eqnmark[colorM]{M}{M}
%            )
%        \big)
%    \end{equation*}
%    \annotatetwo[yshift=1em]{above}{p1}{p2}{\# of nodes}
%    \annotatetwo[yshift=-1em,xshift=0.2ex]{below}{T1}{T2}{\# of graphs in $\hat{\mathcal{G}}_T$}
%    \annotatetwo[yshift=-2em]{below}{k1}{k2}{max.\ indegree in $\hat{\mathcal{G}}_T$}
%    \annotate[yshift=3em]{above,left}{Is}{size of set of allowed interventions}
%    \annotate[yshift=1em]{above}{Nb}{\# of samples per batch}
%    \annotate[yshift=-1em]{below}{M}{\# of samples for $\mathbb{E}_y$}
%    \annotate[yshift=-3em]{below,left}{b}{batch index}
\end{frame}

\end{document}
